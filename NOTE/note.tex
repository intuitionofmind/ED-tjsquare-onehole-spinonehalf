\documentclass[eprint]{article} %{revtex4-1}%
\usepackage{amssymb}  % include amsfonts package
\usepackage{amsfonts}  % American Math Society fonts, define \mathbb, \mathfrak 
\usepackage{amsmath}  % multi-lines formulars, \cfrac
\usepackage{amsthm}  % provide a PROOF enviroment
\usepackage{bm}  % bold math
\usepackage{bbm}  % hollow math
\usepackage[colorlinks=true]{hyperref}
\usepackage{graphicx}
\usepackage{tabularx}
\usepackage{bbding}

\begin{document}

\title{Note}
\author{Wayne Zheng \\ intuitionofmind@gmail.com}
%\homepage[\HandRight\quad]{http://intuitionofmind.bitbucket.org/}
%\affiliation{Institute for Advanced Study, Tsinghua University, Beijing, 100084, China}

%\begin{abstract}
%\end{abstract}

\maketitle
\section{Basic Construction}
The Hamiltonian for $t$-$J$ model is $H_{t\text{-}J}=H_{t}+H_{J}$ where
\begin{equation}\label{tj}
\begin{split}
H_{t} &= -t\sum_{\langle{ij}\rangle, \sigma}(c_{i\sigma}^{\dagger}c_{j\sigma}+h.c.), \\
H_{J} &= J\sum_{\langle{ij}\rangle}\left(\mathbf{S}_{i}\cdot\mathbf{S}_{j}-\frac{1}{4}n_{i}n_{j}\right).
\end{split}  
\end{equation}
Suppose the square lattice is formed with $N_{x}\cdot{N}_{y}=N$ sites and they have been numbered as $0, \cdots, N-1$ in a certain way, for instance, a \emph{snake}. With consideration of one hole doped case, a generic basis can be defined in such a one-dimensional way
\begin{equation}
    c_{0\sigma_{0}}^{\dagger}\cdots{c}_{h-1\sigma_{h-1}}^{\dagger}c_{h+1\sigma_{h+1}}^{\dagger}\cdots{c}_{N-1\sigma_{N-1}}^{\dagger}|0\rangle=(-)^{h}c_{h\sigma_{h}}|s\rangle\equiv|h; s\rangle,
    \label{}
\end{equation}
where $|s\rangle\equiv{c}_{0\sigma_{0}}^{\dagger}\cdots{c}_{N-1\sigma_{N-1}}^{\dagger}|0\rangle$ is the half-filled spin background created by \emph{ordered} fermionic operators. $|h; s\rangle$ thus can be represented as a bosonic configuration in computational program. Here our major task is to compute the vector multiplication required by the package ARPACKPP\cite{arpackpp}. $H_{J}$ can be evaluated as same as the bosonic Heisenberg spin model as one diagonal block of the $H_{t\text{-}J}$ matrix in our representation. For $H_{t}$, we would like to compute the electron's hopping term from site $h$ to site $h{'}$
\begin{equation}
    \begin{aligned}
    &\sum_{\sigma}(c_{h\sigma}^{\dagger}c_{h{'\sigma}}+h.c.)|h; s\rangle \\
    &=c_{h\sigma_{h{'}}}^{\dagger}c_{h{'}\sigma_{h{'}}}(-)^{h}c_{h\sigma_{h}}|s\rangle=c_{h{'}\sigma_{h{'}}}(-)^{h+1}(c_{h\sigma_{h{'}}}^{\dagger}c_{h\sigma_{h}})|s\rangle \\
    &=(-)^{h-h{'}+1}(-)^{h{'}}c_{h{'}\sigma_{h{'}}}|s{'}\rangle.
    \end{aligned}
    \label{}
\end{equation}
Note that what $|s{'}\rangle$ differs from $|s\rangle$ is that the fermionic creation operator $c_{h\sigma_{h}}^{\dagger}$ in $|s\rangle$ is replaced by $c_{h\sigma_{h{'}}}^{\dagger}$ at site $h$. That is to say, in order to evaluate the non-zero matrix elements in terms of $H_{t}$ connecting different bosonic Heisenberg sub-blocks, despite considering the bosonic $|h{'}; s{'\rangle}$ namely result hole's position and spin configuration, an extra fermionic sign $(-)^{h-h{'}+1}$ should be taken in to consideration.

\bibliographystyle{apsrev4-1} 
\bibliography{noteBib}
\end{document} 
